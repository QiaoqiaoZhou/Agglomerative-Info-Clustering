\section{Proofs of main results}
 \label{sec:proof}
 
\begin{Proof}[Theorem~\ref{thm:JT}]
	Consider $`g_N,\mcP_{N-1}$, and $\mcP_{N}$ as in Proposition~\ref{prop:clusters} and let $h_{`g}$
	be as in~\eqref{eq:residualH}.
	By Proposition~\ref{prop:psp},
	%we have $h_{`g_N}[\mcP_{N-1}] = \hat{h}_{`g_{N}}(V) = h_{`g_{N}}[\mcP_{N}]$. Hence,
	as detailed below, we have for all $C\in \mcP_{N-1}`/\mcP_{N}$ that 
	%Then for all $C\in \mcP_{N-1}`/\mcP_{N}$, we have by Proposition~\ref{prop:psp} (applied to $C$) that 
	\begin{subequations}
		\begin{align}
			h_{`g_{N}}(C)&=\sum_{i\in C}h_{`g_{N}}(\Set{i}) \kern1em \text{or equivalently,}
			\label{eq:dt-gn-a}
			\\
			`g_{N}&=J_{T}(\RZ_C).
			\label{eq:dt-gn-b}
		\end{align}
	\end{subequations}
\begin{compactitem}

	\item  The equivalence between~\eqref{eq:dt-gn-a} and~\eqref{eq:dt-gn-b} follows immediately from
		the definition of $J_{\opT}$~\eqref{eq:JT}. More precisely, \eqref{eq:dt-gn-b} is equivalent to
	\begin{alignat*}{2}
	&\kern1em `g_N  =\frac1 {\abs{C}-1}`1[\sum_{i \in C}h(\Set {i})- h(C)`2] &\kern1em &\text {by \eqref{eq:JT}} \\
	&\iff (\abs{C}-1) `g_N = \sum_{i \in C} h(\Set {i})- h(C) & & \because \abs {C}>1\\
	&\iff \kern1.8em h_{`g_N}(C) = \sum_{i \in C} h_{`g_N}(\Set {i}) && \text {by \eqref{eq:residualH}}
	\end{alignat*}
	which is equivalent to \eqref{eq:dt-gn-a} as desired.

	\item ``$\geq$'' for \eqref{eq:dt-gn-a}
	follows from 
	\begin{align*}
	h_{`g_{N}}\big[\Set{C}\cup\Set{\Set{i}\mid i\in V`/C}\big]
	\geq
	h_{`g_{N}}[\mcP_N]
	\end{align*}
	since $\mcP_N$ is optimal to $\hat{h}_{`g_{N}}(V)$~\eqref{eq:DT} by construction.
	(The notation $h_{`g_{N}}[\cdot]$ is defined in~\eqref{eq:DT:2}.)

	\item To explain ``$\leq$'' for \eqref{eq:dt-gn-a}, note that $\mcP_{N-1}$ is an optimal partition to the
		Dilworth truncation~\eqref{eq:DT:1} for $`g\in [`g_{N-1},`g_N)$ by Proposition~\ref{prop:psp}.
			By continuity of $h_{`g}[\mcP_{N-1}]$~\eqref{eq:DT:2} with respect to $`g$, we have that
			$\mcP_{N-1}$ is also optimal for $`g=`g_N$, i.e.,
	\begin{align}
	h_{`g_{N}}[\mcP_N] = h_{`g_{N}}[\mcP_{N-1}],
	\label{eq:dt-gn-pn-pnm1}
	\end{align}
	by the optimality of $\mcP_N$. Hence, since
	\begin{align*}
	\sum\limits_{i\in \mcP_{N}} h_{`g_{N}}(\{i\})  = \sum\limits_{C\in \mcP_{N-1}} \sum\limits_{i\in C} h_{`g_{N}}(\{i\}),
	\end{align*}
	we have,
	\begin{align*}
	\sum_{C\in\mcP_{N-1}`/\mcP_N}`1[h_{`g_{N}}(C)-\sum_{i\in C}h_{`g_{N}}(\Set{i})`2]=0.
	\end{align*}
	Each term in the bracket is non-negative as argued before, and so must be equal to zero.
\end{compactitem}
		
	Consider any maximal solution $C'$ to the r.h.s.\ of~\eqref{eq:maxJT}. Then,
	\begin{subequations}
		\label{eq:`g_N<JT(C')}
		\begin{align}
			`g_{N}& \leq J_{\op{T}}(\RZ_{C'}) \kern1em \text{or equivalently,} \label{eq:gn-leq-jt1}\\ %  \label{eq:`g_N<JT(C'):1}\\
			h_{`g_{N}}(C')& \leq \sum_{i\in C'}h_{`g_{N}}(\Set{i}), \label{eq:gn-leq-jt2} % \label{eq:`g_N<JT(C'):2}
		\end{align}
	\end{subequations}
	which follow from \eqref{eq:dt-gn-b} and \eqref{eq:dt-gn-a}, respectively, since \eqref{eq:maxJT}
	does not require $C'\in \mcP_{N-1}`/\mcP_N$. With
	\begin{align*}
		\mcP':=\Set{C'}\cup\Set{\Set{i}\mid i\in V`/C'},
	\end{align*}
	\eqref{eq:gn-leq-jt2} implies
	\begin{align*}
		h_{`g_N}[\mcP'] \leq h_{`g_N}[\mcP_N],
	\end{align*}
	which must hold with equality by the optimality of $\mcP_N$. Therefore, \eqref{eq:gn-leq-jt1}
	also holds with equality, and so we have \eqref{eq:maxJT} since $`g_N=I^*(\RZ_V)$. Note that, by
	\eqref{eq:dt-gn-pn-pnm1}, $\mcP_{N-1}$ is also an optimal partition. Furthermore, it is the largest/coarsest such
	partition since it is optimal for some $`g<`g_N$ and therefore larger/coarser than all optimal
	partitions for $`g=`g_N$ by Proposition~\ref{prop:plp}. Therefore, the set of maximal optimal
	solutions to the r.h.s. of \eqref{eq:maxJT}  is $\mcP_{N-1}`/\mcP_N$, which is $\pzC^*(\RZ_V)$
	trivially, and so we have \eqref{eq:argmaxJT}.
	
	To prove~\eqref{eq:maxJT:x} and~\eqref{eq:argmaxJT:x}, let $`g^*=I^*(\RZ_V)$. Then, it follows from \eqref{eq:maxJT} that
	%\begin{subequations}
		%\label{eq:I^*:f}
		\begin{align}
			0&=\min_{C\subseteq V:\abs{C}>1}`g^*-J_{\op{T}}(\RZ_C) \nonumber \\
			&\utag{a}=\min_{C\subseteq V:\abs{C}>1}(\abs{C}-1)`g^*+h(C)-\sum_{i\in C}h(\Set{i}) \nonumber \\
			&\utag{b}=\min_{j\in V}\min_{B\subseteq U_j:\abs{B}\geq1} g_j(B)+`g^*\abs{B},
			\label{eq:gjb}
		\end{align}
	%\end{subequations}
	where
\begin{compactitem}
	\item \uref{a} is obtained by the definition of $J_{\opT}$~\eqref{eq:JT} and by multiplying both
		sides of the equality by $\abs{C}-1>0$, which does neither violate the equality nor change the
		set of solutions; and
	\item \uref{b} is obtained by changing the variable $C$ to $(j,B)$ using the bijection that sets
	\begin{align*}
		j:=\min_{i\in C} i
		\kern 1em \text {and}\kern1em B:=C`/\Set{j},
	\end{align*}
	which is possible because $\abs {C}>1$.
	We have also applied $\abs {B}=\abs {C}-1$ and the definition of $g_j$ to rewrite the expression in the minimization. %The set of solutions is preserved in the sense that $C^*$ is optimal to \eqref{eq:JT} iff 
\end{compactitem}
Consider any optimal solution $j^*$ to the r.h.s.\ of \eqref{eq:gjb}.
Then, it follows that $S_{`l}(g_{j^*})$ in~\eqref{eq:S_`l} is such that 
%\begin{align*}
%	S_{`l}(g_{j^*}) 
%	\begin{cases}
%		\utag{c}\neq `0 & `l=-`g^*\\
%		\utag{d}= `0 & `l<-`g^*,
%	\end{cases}
%\end{align*}
\begin{align*}
	\begin{array}{ll}
		S_{`l}(g_{j^*}) 
		 \utag{c}\neq `0, & `l=-`g^* \\
		S_{`l}(g_{j^*}) 
		 \utag{d}= `0, & `l<-`g^*,
	\end{array}
\end{align*}
where
\begin{compactitem}
	\item \uref{c} is because, by \eqref{eq:gjb},
	\begin{align*}
		`0&\neq \max \argmin_{B\subseteq U_{j^*}:\abs{B}\geq 1} g_{j^*}(B)+`g^*\abs{B}\\
		&= \max \argmin_{B\subseteq U_{j^*}} g_{j^*}(B)+`g^*\abs{B}\\
		&=S_{-`g^*}(g_{j^*}).
	\end{align*}
	The last equality is by the definition~\eqref{eq:S_`l} of $S_{-`g^*}$. The first equality is because allowing $B=`0$ does not change the minimum value $0$, since
	\begin{align*}
		g_{j^*}(`0)+`g^*\abs {`0}=0
	\end{align*}
	 as $g_{j^*}(`0)$. Doing so also does not affect the maximum minimizer since the new optimal solution introduced, namely $`0$, cannot be maximum trivially.
	\item To explain \uref{d}, note that \eqref{eq:gjb} implies for all $B\subseteq U_{j^*}:\abs {B}>1$ that
	\begin{alignat*}{2}
		g_{j^*}(B)+`g^*\abs{B} &\geq 0&\kern1em& \text {and so}\\
		g_{j^*}(B)-`l\abs{B} &\geq (-`l-`g^*) \abs {B} && \forall `l\in `R\\
		& > 0 && \forall `l<-`g^*.
	\end{alignat*}
	In other words, for $`l<-`g^*$, we have that $`0$ is the unique solution to $\min_{B\subseteq U_{j^*}} g_{j^*}(B)+`g^*\abs{B}$, which implies \uref{d} by the definition~\eqref{eq:S_`l} of $S_{`l}$.
\end{compactitem}
Now, \uref{c} and \uref{d} imply that
\begin{align}
	-`g^*&=\sup \Set{`l\in `R\mid S_{`l}(g_{j^*})=`0} \nonumber \\
	&= \min_{i\in U_{j^*}} \min_{`l\in `R:i\in S_{`l}(g_{j^*})} `l \nonumber \\
	&= \min_{i\in U_{j^*}} x^{(j^*)}_i
	\label{eq:gamma-xi}
\end{align}
where the last equality is by \eqref{eq:MNB:x*} since $x^{(j^*)}_{U_{j^*}}$ denotes the minimum norm base for $g_{j^*}$. The equality implies \eqref{eq:maxJT:x} as desired.

To prove \eqref{eq:argmaxJT:x}, consider any set $C^*$ that belongs to the r.h.s.\ of \eqref{eq:argmaxJT}. Applying the bijection
	\begin{align*}
	j^*:=\min_{i\in C^*} i
	\kern 1em \text {and}\kern1em B^*:=C^*`/\Set{j^*},
\end{align*}
$(j^*,B^*)$ is a solution to the r.h.s.\ of \eqref{eq:gjb}. By the inclusion-wise maximality of $C^*$, the set $B^*$ is also a maximal (the maximum) solution, i.e.,
\begin{align*}
	B^*&\utag{e}= S_{-`g^*}(g_{j^*})\\
	&= \Set {i\in U_{j^*}\mid i\in S_{-`g^*}(g_{j^*}) }\\
	&\utag{f}= \Set {i\in U_{j^*} \mid x^{(j^*)}_i \leq -`g^*}\\
	&\utag{j}= \argmin_{i\in U_{j^*}} x^{(j^*)}_i,
\end{align*}
where \uref{e} is by \eqref{eq:S_`l}; \uref{f} is by \eqref{eq:MNB:S_`l}; and \uref{j} is by
\eqref{eq:gamma-xi}. Hence, $C^*$ also belongs to the r.h.s.\ of \eqref{eq:argmaxJT:x}. 

Conversely, if $(j^*,B^*)$ is an optimal solution to the r.h.s.\ of \eqref{eq:gjb}, it can be argued
easily that $C^*:=\Set{j}^*\cup B^*$ is also a solution to the r.h.s.\ of \eqref{eq:maxJT}. The
maximal such $C^*$ therefore belongs to the l.h.s.\ of \eqref{eq:argmaxJT:x} as desired. This
completes the proof.
\end{Proof}


\begin{Proof}[Theorem~\ref{thm:`g:P:fused}]
	First, we prove \eqref{eq:A:`g_l} and \eqref{eq:A:P_i-1} using the general
	property~\eqref{eq:imunion} instead of the precise definition~\eqref{eq:mmi} of the MMI. 
	%
	
	We first argue that, for any feasible solution $\mcF$ to the r.h.s.\ of~\eqref{eq:A:`g_l}, 
	\begin{align}
		\label{eq:mmi-zf-upper}
		I(\RZ_{\bigcup\mcF}) \leq `g_\ell.
	\end{align}
	Suppose to the contrary that $I(\RZ_{\bigcup\mcF})>`g_{\ell}$. Then, there exists
	$C\supseteq\bigcup \mcF$ such that
	$C\in \pzC_{`g_\ell}(\RZ_V)$ by the definition~\eqref{eq:clusters} of clusters. However,
	$\pzC_{`g_\ell}(\RZ_V)\subseteq \mcP_{\ell}$ by Proposition~\ref{prop:clusters}, which
	contradicts the fact that $\mcF\subseteq \mcP_{\ell}$ with $\abs {\mcF}>1$. 
	
	Next, we show that \eqref{eq:mmi-zf-upper} can be achieved with equality for some feasible solution $\mcF$.
	Consider any $C\in \mcP_{\ell-1}`/\mcP_{\ell}$. (Such a $C$ exists since $\mcP_{\ell}$ is
	strictly finer than $\mcP_{\ell-1}$.) Then, we have 
	\begin{align}
	C
	%\utag{b}
	= \bigcup \mcF\text{ for some }\mcF \subseteq \mcP_{\ell}:|\mcF| > 1,
	\label{eq:c-feasible}
	\end{align}
	i.e., for some feasible solution $\mcF$.
	By Proposition~\ref{prop:clusters}, we have 
	\begin{align*}
		C\in \pzC_{`g}(\RZ_V)\kern1em \text{ for all
		 }`g\in [`g_{\ell-1},`g_\ell),
	\end{align*}
	and so $I(\RZ_C)\geq `g_\ell$, i.e., larger than all values in the interval.
	The reverse inequality also holds by \eqref{eq:mmi-zf-upper} and \eqref{eq:c-feasible}. Hence, we have
	\begin{align}
		I(\RZ_C)
		%\utag{c}
		=`g_\ell,
		\label{eq:mmi-zc-equal}
	\end{align}
	 which implies \eqref{eq:A:`g_l} as desired. 
	
	Now, we argue that the above construction gives all the optimal solutions to the r.h.s.\ of
	\eqref{eq:A:`g_l}, hence establishing ~\eqref{eq:A:P_i-1}. For any $C\in
	\mcP_{\ell-1}`/\mcP_{\ell}$, \eqref{eq:c-feasible} and \eqref{eq:mmi-zc-equal} imply that ``$\subseteq$'' holds
	for~\eqref{eq:A:P_i-1}, because 
	the fact that $C\in \pzC_{`g_{\ell-1}}(\RZ_V)$ (by Proposition~\ref{prop:clusters}) means that it
	is maximal by the definition~\eqref{eq:clusters} of  clusters. 
	
	To argue the reverse inclusion ``$\supseteq$'', consider any $\mcF$ belonging to the r.h.s.\
	of~\eqref{eq:A:P_i-1}. By \eqref{eq:A:`g_l}, 
	\begin{align*}
		I(\RZ_{\bigcup \mcF})=`g_\ell>`g_{\ell-1}
	\end{align*}
	and so $\bigcup \mcF\in \pzC_{`g_{\ell-1}}$ by the definition~\eqref{eq:clusters} of clusters and
	the maximality of $\bigcup \mcF$. This completes the poof of~\eqref{eq:A:P_i-1}.
	
	Consider proving \eqref{eq:A:`g_l:fused}
	and~\eqref{eq:A:P_i-1:fused}.
	For any optimal solution $\mcF$ to \eqref{eq:A:`g_l},  we have
	\begin{align*}
		\mcP^*(\RZ_{\bigcup \mcF})=\mcF=\Set{C_j\mid j\in B}
		%\label{eq:a:4}
	\end{align*}
	for some $B\subseteq U$, 
	where the first equality is by Proposition~\ref{prop:fund:cluster}
	since $I(\RZ_C)>`g_\ell$ for all $C\in \mcF$ such that ${C}>1$. Hence,
	\begin{align*}
		I(\RZ_{\bigcup \mcF})=I_{\mcP^*(\RZ_{\bigcup \mcF})}(\RZ_{\bigcup \mcF})&=I_{\mcF}(\RZ_{\bigcup \mcF})\\&=I_{\Set{\Set{j}\mid j\in B}}(\RZ'_B)\geq I(\RZ'_B)
	\end{align*}
	where the inequality follows from the fact that partition $\Set{\Set{j}\mid j\in B}$ into
	singletons  may not be the optimal partition of $B$ for $I(\RZ'_B)$. The reverse inequality also
	holds because, for all $\mcP'\in \Pi'(B)$, define 
	\begin{align*}
		\mcP:=\Bigg\{\bigcup_{j\in C'}C_j\mid C'\in\mcP' \Bigg\}\in \Pi'\Big(\bigcup\mcF\Big),
	\end{align*}
	we have $I_{\mcP'}(\RZ'_B)=I_{\mcP}(\RZ_{\bigcup\mcF})\geq I(\RZ_{\bigcup \mcF})$. Here, the
	inequality follows from the fact that $\mcP$ may not be the optimal partition of
	$\RZ_{\bigcup\mcF}$.  This completes the proof of Theorem~\ref{thm:`g:P:fused}.
\end{Proof} 


