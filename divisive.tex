\section{Divisive approach}
\label{sec:divisive}

With the laminar structure in Proposition~\ref{prop:clusters}, a divisive algorithm was given in \cite{chan16cluster} to
compute $`g_\ell$ and $\mcP_{\ell}$ by iteratively producing finer partitions from coarser ones,
%i.e., from $\ell=1$ to $N$. 
i.e., from $\mcP_1$ to $\mcP_N$. 
The following result is instrumental in the devising of such an algorithm.
%
\begin{Proposition}[\mbox{\cite[Theorems~5.2 \& 5.3]{chan15mi}, \cite[Theorem~4]{chan16cluster}}]
	\label{prop:fund:cluster}
	The optimal partitions achieving the MMI in~\eqref{eq:mmi} together with the trivial partition $\{V\}$ form a
	lattice w.r.t.\ \eqref{eq:finer}. The minimum/finest optimal partition,
	%denoted  $\mcP^{*}(\RZ_V)$, satisfies 
	called the fundamental partition and denoted by $\mcP^{*}(\RZ_V)$, satisfies 
	\begin{align*}
		\mcP^{*}(\RZ_V) \backslash \{\{i\}\mid i\in V\} = \pzC_{I(\RZ_V)}(\RZ_V),
	\end{align*}
	with $\pzC_{`g}$ defined in \eqref{eq:clusters}.
	In other words,  $`g_1 = I(\RZ_V)$ and $\mcP_{1} = \mcP^{*}(\RZ_V)$ in
	Proposition~\ref{prop:clusters}. Finally, for $\ell \geq 1$,
	%if there exists $C \in \mcP_{\ell}:\abs {C}>1$,
	%if $\mcP_{\ell}$ is not the singletons partition,
	if $\mcP_l$ is not the partition into singletons,
	then $`g_{\ell +1} = \min\nolimits_{C\in \mcP_{\ell}:\abs{C}> 1}I(\RZ_C)$ and
	$\mcP_{\ell+1}\supseteq \mcP^*(\RZ_C)$ for any minimizer $C$.
\end{Proposition}
%Together with the laminar structure in Proposition~\ref{prop:clusters}, it can be argued that 
Because of the last statement in the proposition, any algorithm for computing $\mcP^{*}(\RZ_V)$ can
be applied iteratively for the divisive info-clustering approach as in~\cite[Algorithm~2]{chan16cluster}.
However,
as discussed near the end of~Section~\ref{sec:introduction},
this divisive approach can be a factor $n$ slower than \cite[Algorithm~3]{chan16cluster}.

\begin{example}
	\label{eg:divisive}
	As an illustration of Proposition~\ref{prop:fund:cluster} and the divisive approach, consider
	$\RZ_{\{1,\dots,6\}}$ in \eqref{eq:eg-motivate}.
%%%%%%%%%%%%%%%%%%%%%%%%%%%%%%%%%%%%%%%%%
	The finest optimal partition
	$\mcP^{*}(\RZ_V) = \Set{\Set{1,2,3},\Set{4,5},\Set{6}}$ is shown in \figref{fig:eg-div-zv}, and
	so one can easily verify from \eqref{eq:I} that $I(\RZ_{V}) = 0$.
	For completion, the figure also shows the lattice of optimal partitions stated in the proposition,
	where the trivial partition $\Set{V}$ is indicated using a dashed line.
	Similarly, \figref{fig:eg-div-z123} shows the optimal partitions of
	$\RZ_{\Set{1,2,3}}$.
	Since $I(\RZ_{V}) = 0$, Proposition~\ref{prop:fund:cluster}
	asserts that $\pzC_{0}(\RZ_V) = \Set{\Set{1,2,3},\Set{4,5}}$.
	Since $I(\RZ_{\Set{1,2,3}}) =I(\RZ_{\Set{4,5}})=1$ (and $\Set{1,2,3}$
	is the only cluster
	%among the two whose finest optimal partition is not the partition into singletons),
	%whose finest optimal partition contains a non-singleton),
	with a non-singleton in its finest optimal partition),
	Proposition~\ref{prop:fund:cluster} asserts that
	$\pzC_{1}(\RZ_V)=\pzC_1(\RZ_{\Set {1,2,3}}) = \Set{\Set{1,2}}$. 

%%%%%%%%%%%%%%%%%%%%%%%%%%%%%%%%%%%%%%%%%
%	Assuming an algorithm for computing the
%	finest optimal partition, the divisive algorithm starts by computing 
%	$\mcP^{*}(\RZ_V) = \Set{\Set{1,2,3},\Set{4,5},\Set{6}}$ (from which $I(\RZ_V)$ is readily
%	available), declares the non-singleton elements as clusters at threshold $I(\RZ_V)$,
%	and proceeds iteratively by picking any cluster (of size larger than two) and computing its
%	finest optimal partition, etc. In our example, there are two clusters, the sets $\Set{1,2,3}$ and $\Set{4,5}$
%	at threshold $0$.
%%	The finest optimal partition of $\RZ_{\Set{1,2,3}}$ is shown in
%%	Fig.~\ref{fig:eg-div-z123}, which results in the cluster $\Set{1,2}$ at threshold
%%	$I(\RZ_{\Set{1,2,3}}) = 1$.
%%	Also, $I(\RZ_{\Set{4,5}}) = 1$ with $\mcP^{*}(\RZ_{\Set{4,5}}) = \Set{\Set{4},\Set{5}}$.
%	%
%	At threshold $I(\RZ_{\Set{1,2,3}}) = 1$, $\Set{1,2,3}$ results in the cluster $\Set{1,2}$ since this is only nonsinglton element of $\mcP^{*}(\RZ_{\Set{1,2,3}})$ as shown in Fig.~\ref{fig:eg-div-z123}.
%	At threshold $I(\RZ_{\Set{4,5}}) = 1$, $\Set{4,5}$ gives no clusters since $\mcP^{*}(\RZ_{\Set{4,5}}) = \Set{\Set{4},\Set{5}}$.
%	At threshold $I(\RZ_{\Set{1,2}}) = 2$, $\Set{1,2}$ gives no clusters since $\mcP^{*}(\RZ_{\Set{1,2}}) = \Set{\Set{1},\Set{2}}$.
%	After this, the algorithm terminates with the entire hierarchy of clusters as in \eqref{eq:eg-clusters}.
%	%
	Assuming an algorithm for computing the
	finest optimal partition, the divisive algorithm starts by computing 
	$\mcP^{*}(\RZ_V) = \Set{\Set{1,2,3},\Set{4,5},\Set{6}}$ (from which $I(\RZ_V)$ is readily
	available), declares the non-singleton elements as clusters at threshold $I(\RZ_V)$,
	and proceeds iteratively by picking any cluster (of size larger than two) and computing its
	finest optimal partition, etc.
	In our example, $\mcP^{*}(\RZ_V)$ is shown in Fig.~\ref{fig:eg-div-zv}, which results in the clusters $\Set{\Set{1,2,3}, \Set{4,5}}$ at threshold $I(\RZ_{V}) = 0$.
	Next, $\mcP^{*}(\RZ_{\Set{1,2,3}})$ is shown in Fig.~\ref{fig:eg-div-z123}, which results in the cluster $\Set{1,2}$ at threshold $I(\RZ_{\Set{1,2,3}}) = 1$.
	%$\mcP^{*}(\RZ_{\Set{4,5}}) = \Set{\Set{4},\Set{5}}$, which procudces no clusters at threshold $I(\RZ_{\Set{4,5}}) = 1$.
	%$\mcP^{*}(\RZ_{\Set{1,2}}) = \Set{\Set{1},\Set{2}}$, which prodices no clusters at threshold $I(\RZ_{\Set{1,2}}) = 2$.
	After this, the divisive algorithm terminates with the entire hierarchy of clusters as in \eqref{eq:eg-clusters}.
	(In this example, the algorithm did not compute the last threshold in \eqref{eq:eg-clusters}, but it can do
	so by letting it proceed to clusters of size two.)
\end{example}
%%%%%%%%%%%%%%%%%%%%%%%%%%%%%%%%%

%The divisive approach can be a factor $n$ slower than \cite[Algorithm~3]{chan16cluster} as discussed in Section~\ref{sec:complextiy}.
%The divisive approach can be a factor $n$ slower than \cite[Algorithm~3]{chan16cluster} that
%computes (in no particular order) all the clusters. %Nevertheless, in practice it is desirable to have 
%Nevertheless, an iterative approach of computing the clusters in order remains desirable
%in practice as it can stop when further computations are not of
%interest or are meaningless due to errors in estimating the entropies from data.
%Agglomerative info-clustering will achieve this appealing iterative computation (according to order)
%while retaining the same run-time complexity as \cite[Algorithm~3]{chan16cluster}. 


