\section{Principal Squence of Partitions and Dilworth Truncation}
\label{sec:dt}
%The ability to compute info-clustering solution efficiently stems from the submodularity \eqref{eq:submodular} of
%the entropy function.
%In combinatorial optimization~\cite{schrijver02}, submodularity is well-known to give rise to polynomial-time solutions. 
%The info-clustering problem, in particular, relies on the following closely related polynomial-time solvable structures.

%\subsection{Principal sequence of partitions}
\label{sec:PSP}
%\input{PSP}
In this section we give a brief discussion on the PSP and the  Dilworth truncation,
which will be usefull for proving the main results in Appendix~\ref{sec:proof}.
%and an  equvalent characterization of the MMI \eqref{eq:mmi}.
For any real number $`g\in `R$, define the \emph{residual entropy function}~\cite{chan15mi} of a random vector $\RZ_V$ as
\begin{align}
	h_{`g}(B)&:= h(B)-`g \kern2.5em \text{for $B\subseteq V$}, \label{eq:residualH}
\end{align}
where $h$ is the entropy function \eqref{eq:h}.
The residual entropy function is also submodular and its \emph{Dilworth truncation} evaluated at $V$ is defined as~\cite{schrijver02}
\begin{subequations}
	\label{eq:DT}
\begin{align}
		\hat{h}_{`g}(V)&:= \min_{\mcP\in \Pi(V)} h_{`g}[\mcP] \kern1em\text{where}\label{eq:DT:1}\\
		h_{`g}[\mcP] &:= \sum\nolimits_{C\in \mcP} h_{`g}(C). \label{eq:DT:2} 
\end{align} 
\end{subequations}
This definition provides an alternative charaterization of the MMI as the smallest $`g$ such that
\eqref{eq:DT:1} holds for some $\mcP \in \Pi'(V)$ \cite{chan15mi}. Morevoer, we have the
following structural property of optimal partitions to \eqref{eq:DT:1}.
\begin{Proposition}[\mbox{\cite{narayanan90}}]
	\label{prop:plp}
	Submodularity~\eqref{eq:submodular} of $h$~\eqref{eq:h} implies that the set of optimal
	partitions to the Dilworth truncation~\eqref{eq:DT:1} forms a lattice, called the Dilworth
	truncation lattice, with respect to the partial order~\eqref{eq:finer}. Furthermore, if $\mcP'$
	and $\mcP''$ are the optimal partitions for $`g'$ and $`g''$ respectively, then $`g'<`g''$
	implies $\mcP'\succeq \mcP''$.
\end{Proposition}
In particular, the minimum/finest optimal partition exists and characterizes the info-clustering solution as follows:
% the corresponding threshold. %and $\Pi(V)$ is the collection of all partitions of $V$ into non-empty subsets.
\begin{Proposition}[\mbox{\cite[Corollary~2]{chan16cluster}}]%[\mbox{\cite[Corollary~3.1]{chan16cluster}}]
	\label{prop:psp}
	For a finite set $V$ with size $\abs{V}>1$ and a random vector $\RZ_V$,
	\begin{align}
		\begin{split}
	\pzC_{`g}(\RZ_V)
	&=`1[\min \Set{\mcP\in \Pi(V) \mid h_{`g}[\mcP]=\hat{h}_{`g}(V)}`2]\\ &\kern4em \big\backslash`1\{\Set{i}\mid i\in V`2\}, 
	   \end{split}
	\end{align}
namely, the non-singleton elements of the finest optimal partition to the Dilworth truncation~\eqref{eq:DT:1}.
	%where $\hat{h}_{`g}$ is the Dilworth truncation~\eqref{eq:DT:2} of the residual entropy function~\eqref{eq:residualH} of $\RZ_V$. 
\end{Proposition}

Hence, in Proposition~\ref{prop:clusters}, the sequence of $\mcP_{\ell}$ for $\ell$ from $1$ to $N$
with the corresponding $`g_\ell$ also characterizes the minimum optimal partitions
to~\eqref{eq:DT:1} for all $`g\in `R$. This sequece of partitions (together with the critical values) is known as the \emph{principal sequence of partitions} (PSP) of
$h$ and was introduced in~\cite{narayanan90}.
In other words,
%The sequence of partitions in the proposition (together with the critical values) was further shown in
%\cite[Corollary~2]{chan16cluster} to be the \emph{principal sequence of partitions} (PSP) \cite{narayanan90} of the
%entropy function of $\RZ_{V}$. (We will further discuss this shortly.)
%%%%%%%%%%%%%%%%%%%%%%%%%%%%%%%%%%%%%%%%%%%%%%%%%%%%%
%From Proposition~\ref{prop:clusters},
given the PSP of the entropy function, one can readily obtain the clustering solution, and vice versa.

%%%%%%%%%%%%%%%%%%%%%%%%%%%%
\begin{figure}
	\begin{center}
%		\subcaptionbox{PLP \label{fig:eg-plp}}{
%			\input{dtl/plp.tex}
%		}%\hfill
		\subcaptionbox{$\hat{h}_{`g}(V)$ \label{fig:eg-dt-fn}}{
			\hspace{-.8cm}
			{\def\u{1.2}
				\def\sx{0.9}
				\def\left{left}
				\def\right{right}
				\def\plabel{-4}
				\tikzstyle{point}=[draw,solid,red,thick,circle,minimum size=.2em,inner sep=.0em, outer sep=.2em]
				\begin{tikzpicture}[remember picture,x=1.4em,y=1.4em,>=latex, every node/.style={font=\scriptsize}]
				\draw[->] (0,-6.5*\u) -- (0,6.5*\u) node (y) [label=right:$\hat{h}_{`g}(V)$] {};
				\draw[->] (-1*\sx*\u,0) -- (3*\sx*\u,0) node [label=below:$`g$] {};
				\foreach \i/\ya/\xa/\yb/\xb/\lp/\ld/\lb in {
					1/5.0/-1.0*\sx/4/0*\sx/above \left/.7em/{$\kern0em h_{`g}[\Set{\Set{1,\dots,6}}]=4-`g$}, 
					2/4/0*\sx/1/1*\sx/\left/0em/{$h_{`g}[\Set{\Set{1,2,3},\Set{4,5},\Set{6}}]=4-3`g$},
					3/1/1*\sx/-4/2*\sx/\left/0em/{$h_{`g}[\Set{\Set{1,2},\Set{3},\Set{4},\Set{5},\Set{6}}]=6-5`g$},
					4/-4/2*\sx/-6.5/2.35*\sx/below \left/0em/{$h_{`g}[\Set{\Set{1},\Set{2},\Set{3},\Set{4},\Set{5},\Set{6}}]=8-6`g$}    
					%4/-4/2/-7/2.5/left/{$h_{`g}[\Set{\Set{1},\Set{2},\Set{3},\Set{4},\Set{5},\Set{6}}]=8-6`g$}    
				}
				\draw[dashed] (\xa*\u,\ya*\u)  to node [inner sep=0,outer sep=0,label={[label distance=\ld]\lp:{\scriptsize\lb}}] {}  (\xb*\u,\yb*\u);
				%\path (0*\sx*\u,4*\u) node (1) [point,red,thick,label=\right:{\scriptsize$p_1$}] {};
				%\path (1*\sx*\u,1*\u) node (2) [point,red,thick,label={[label distance=0em]\right:{\scriptsize$p_2$}}] {};
				%\path (2*\sx*\u,-4*\u) node (3) [point,red,thick,label=\right:{\scriptsize$p_3$}] {};
				\draw[dashed,->](\plabel*\sx*\u,4*\u) node[below]{$0=I(\RZ_{\Set{1,\dots,6}})$} -- (0*\sx*\u,4*\u)node(1)[point]{} -- (0*\sx*\u,0);
				\draw[dashed,->](\plabel*\sx*\u,1*\u) node[above]{$1=I(\RZ_{\Set{1,2,3}})$}node[below]{$=I(\RZ_{\Set{4.5}})$} -- (1*\sx*\u,1*\u)node(2)[point]{} -- (1*\sx*\u,0*\u);
				\draw[dashed,->](\plabel*\sx*\u,-4*\u) node[above]{$2=I(\RZ_{\Set{1,2}})$} -- (2*\sx*\u,-4*\u) node(3)[point]{}-- (2*\sx*\u,0);
				\draw[-,thick,blue] (-1*\sx*\u,5*\u)--(1)--(2)--(3)--(2.35*\sx*\u,-6.5*\u);
				\end{tikzpicture}}
		}%\hfill
		\subcaptionbox{PSP \label{fig:eg-psp-dil}}{
			\def\thickness{very thick}
			\begin{tikzpicture}[remember picture,rotate=0,
			group/.style={fill opacity=.2, inner sep=0, outer sep=0, rounded corners},
			every node/.style={rounded corners, text opacity=1, transform shape}
			]
			\def\position{below}
			\matrix(p0)at(0,0)[matrix of math nodes, ampersand replacement=\&,
			row sep=1mm,
			column sep=.8mm,
			every cell/.style={anchor=base west}]{
				1 \& 2\& 3 \\
				4 \& 5\& 6 \\
				\\};
			\def\dist{1.8}
			\def\distx{.3cm}
			\def\disty{.8cm}
			%
			\matrix(p1) [\position = 0.9*\disty of p0,
			matrix of math nodes, ampersand replacement=\&,
			row sep=1mm,
			column sep=.8mm,
			every cell/.style={anchor=base west}]{
				1 \& 2\& 3 \\
				4 \& 5\& 6 \\
				\\};
			\matrix(p2)[\position = 1.1*\disty of p1,
			matrix of math nodes, ampersand replacement=\&,
			row sep=1mm,
			column sep=.8mm,
			every cell/.style={anchor=base west}]{
				1 \& 2\& 3 \\
				4 \& 5\& 6 \\
				\\};
			\matrix(p3)[\position = 1.1*\disty of p2,
			matrix of math nodes, ampersand replacement=\&,
			row sep=1mm,
			column sep=.8mm,
			every cell/.style={anchor=base west}]{
				1 \& 2\& 3 \\
				4 \& 5\& 6 \\
				\\};
			% clustering of Z in PLP
			\node(p0)[draw, group, fill=none, fit=(p0-1-1)(p0-2-3)]{};
			%
			\node[draw, \thickness, group, fill=none, fit=(p1-1-1)(p1-1-3)]{};
			\node[draw, \thickness, group, fill=none, fit=(p1-2-1)(p1-2-2)]{};
			\node[draw, group, fill=none, fit=(p1-2-3)(p1-2-3)]{};
			\node(p1)[group, fill=none, fit=(p1-1-1)(p1-2-3)]{};
			%
			\node[draw, group, \thickness, fill=none, fit=(p2-1-1)(p2-1-2)]{}; \node[draw, group, fill=none, fit=(p2-1-3)]{};
			\node[draw, group, fill=none, fit=(p2-2-1)]{};
			\node[draw, group, fill=none, fit=(p2-2-2)]{};
			\node[draw, group, fill=none, fit=(p2-2-3)]{};
			\node(p2)[group, fill=none, fit=(p2-1-1)(p2-2-3)]{};
			%
			\node[draw, group, fill=none, fit=(p3-1-1)]{};
			\node[draw, group, fill=none, fit=(p3-1-2)]{};
			\node[draw, group, fill=none, fit=(p3-1-3)]{};
			\node[draw, group, fill=none, fit=(p3-2-1)]{};
			\node[draw, group, fill=none, fit=(p3-2-2)]{};
			\node[draw, group, fill=none, fit=(p3-2-3)]{};
			\node(p3)[group, fill=none, fit=(p3-1-1)(p3-2-3)]{};
			%%%%%%%%%%%%%%%%%%%%%%%%%%%%%%%%%%%%%%%%%%%
			%
			%
			%	% clustering of Z in PLP
			%
			%	\node[draw, group, fill=none, fit=(p3-1-1)]{};
			%	\node[draw, group, fill=none, fit=(p3-1-2)]{};
			%	\node[draw, group, fill=none, fit=(p3-1-3)]{};
			%	\node[draw, group, fill=none, fit=(p3-2-1)]{};
			%	\node[draw, group, fill=none, fit=(p3-2-2)]{};
			%	\node[draw, group, fill=none, fit=(p3-2-3)]{};
			%	%
			\draw(p0)--(p1)--(p2)--(p3);
			\draw[dashed,overlay] (1)--(p0.east|-1);	
			\draw[dashed,overlay] (2)--(p0.east|-2);
			\draw[dashed,overlay] (3)--(p0.east|-3);
			\end{tikzpicture}
		}\hfill
	\end{center}
	\caption{Dilworth truncation}
\label{fig:eg-dt}
\end{figure}

The following example demonstrates the Dilworth truncation for the RVs $\RZ_V$ in
Example~\ref{eg:psp}, where the PSP in Fig.~\ref{fig:eg-psp} is shown again in
Fig.~\ref{fig:eg-psp-dil} for convenience.
\begin{example}
%	\figref{fig:eg-dt}	
	For $\RZ_{\Set{1,\dots,6}}$ in \eqref{eq:eg-motivate}, the Dillworth truncation $\hat{h}_{`g}(V)$ \eqref{eq:DT:1} is shown
	in \figref{fig:eg-dt-fn}. The Dillowrth truncation is piecewise linear with at most $|V|$
	turning points. A turning point $p_i:=(`g_i,\hat{h}_{`g_i}(V))$ occurs when \eqref{eq:DT:1} has
	more than one solution. The collection of such optimal solutions is the Dillowrth truncation
	lattice at $`g_i$ indicated in Proposition~\ref{prop:plp}.%
	\footnote{The Dillowrth truncation lattice is
	not shown in \figref{fig:eg-dt}. This lattice is shown in Figs.~\ref{fig:eg-div-zv} and
	\ref{fig:eg-div-z123} for $\RZ_V$ and $\RZ_{\Set{1,2,3}}$, respectively.}
	The finest optimal partition at $`g_i$ remains (while all other
	partitions seieze to be) optimal until the next turning point. In other words, the finest optimal
	partition at $`g_i$ determines the line segment that follows $`g_i$. The sequence of the finest
	optimal partitions is the PSP, whch is shown in \figref{fig:eg-psp-dil}. 
%	each line segment
%	in the 
%	Moreove, for each 
%	lattices for all values of $`g$ are shown in
%	\figref{fig:eg-plp}. For instance, the sublattice consisting of the top five partitions in
%	\figref{fig:eg-plp} is the Dillworth trunction lattice at $`g=0$, i.e., it contains all the
%	solutions to $h_{0}[\mcP] = \hat{h}_{0}(V)$. (This can be seen from \figref{fig:eg-dtl-fn} at
%	$p_1$.) The finest 
\end{example}
%%%%%%%%%%%%%%%%%%%%%%%%%%%


