\section{Principal sequence and minimum norm base}
\label{sec:prelim}
%\input{prelim}


%The ability to compute the info-clustering solution efficiently stems from the submodularity of
%entropy~\cite{fujishige78}, or equivalently the fact that conditional mutual information is
%non-negative~\cite{shannon48}. More precisely,
%the \emph{submodularity} of entropy means that the entropy function $h$ in \eqref{eq:h} satisfies
%\begin{align}
%	h(B_1)+h(B_2)\geq h(B_1\cup B_2)+h(B_1\cap B_2)  \label{eq:submodular}
%\end{align}
%for all $B_1,B_2\subseteq V$.
%The entropy function is also said to be \emph{normalized} since $h(`0)=0$, and non-decreasing since $h(B')\leq h(B)$ whenever $B' \subseteq B \subseteq V$.
%In combinatorial optimization~\cite{schrijver02}, submodularity is well-known to often give rise to polynomial-time solutions. 
As discussed earlier, the PSP of $h$ is polynomial time solvable, which
characterizes the info-clustering solution by Proposition~\ref{prop:clusters}.
Our agglomerative algorithm
will make use of yet another polynomial time solvable structure, discussed below,
that is closely related to the PSP.
% as explained in Section~\ref{sec:results}.
%The details may be skipped for the first reading.


Consider any submodular function $f:2^U\to `R$~\eqref{eq:submodular} on the finite ground set $U$. For $`l\in `R$, define
\begin{align}
\kern-.5em S_{`l}(f):=\max \argmin_{B\subseteq U} f(B)-`l\abs{B}, \label{eq:S_`l}
\end{align}
where $\max$ is the inclusion-wise maximum.
Note that the minimization in \eqref{eq:S_`l} is a \emph{submodular function minimization (SFM)}
since the function $B\subseteq U\mapsto f(B)-`l\abs{B}$ is also submodular. It is well-known that
the minimizers of a submodular function form a lattice w.r.t. set inclusion~\cite{schrijver02}, and so the maximum
in \eqref{eq:S_`l} exists and is unique.  
\begin{Proposition}[\mbox{\cite{fujishige80,fujishige-pp-revisited}}] 
	\label{prop:ps}
	$S_{`l}(f)$ for $`l\in`R$ satisfies
	 \begin{align*}
		 \label{eq:aaa-ps}
	 	S_{`l'}(f)\subseteq S_{`l''}(f) \kern1em \text {iff} \kern1em `l'\leq `l''.
	 \end{align*}
%	 %and is referred to as the principal sequence (PS).
	 This defines a  sequence, for some positive integer $N'$, 
	 \begin{align}
		 \label{eq:aaa-pss}
		 S_{\lambda_{0}} := \emptyset \subsetneq S_{\lambda_1} \subsetneq \dots \subsetneq S_{\lambda_{N'-1}} \subsetneq S_{\lambda_{N'}} := V
	 \end{align}
	 of distinct sets, which
	 (together with the sequence $-\infty = \lambda_{0} < \lambda_1 < \dots < \lambda_{N'} <
	 \lambda_{N'+1} := \infty$ of $\lambda$ values at which $S_{\lambda}$ changes)
	 is referred to as the principal sequence (PS).
%%	 The sequence 
%%	 \begin{align*}
%%		 S_{\lambda_{0}} := \emptyset \subsetneq S_{\lambda_1} \subsetneq \dots \subsetneq S_{\lambda_{N'-1}} \subsetneq S_{\lambda_{N'}} := V
%%	 \end{align*}
%%	 of distinct sets
%%	 (together with the sequence $-\infty = \lambda_{0} < \lambda_1 < \dots < \lambda_{N'} < \lambda_{N'+1} := \infty$)
%%	 is referred to as the principal sequence (PS).
\end{Proposition}
%$S_{`l}(f)$ can be characterized and computed in strongly polynomial time as follows:
%	Equation \eqref{eq:aaa-ps} defines a  sequence, for some positive integer $N'$, 
%	 \begin{align*}
%		 S_{\lambda_{0}} := \emptyset \subsetneq S_{\lambda_1} \subsetneq \dots \subsetneq S_{\lambda_{N'-1}} \subsetneq S_{\lambda_{N'}} := V
%	 \end{align*}
%	 of distinct sets. This sequence
%	 (together with the sequence $-\infty = \lambda_{0} < \lambda_1 < \dots < \lambda_{N'} <
%	 \lambda_{N'+1} := \infty$ of $\lambda$ values at which $S_{\lambda}$ changes)
%	 is referred to as the principal sequence (PS).

W.l.o.g., we assume $f$ is normalized, i.e., $f(`0)=0$, since we can redefine
$f$ as $f-f(`0)$ without  affecting the solutions to the SFM in~\eqref{eq:S_`l}, i.e., aside from a
shift in the $\lambda_i$ values, the
PS~\eqref{eq:aaa-pss} is
invariant to a constant shift in the submodular function. The polyhedron $\rmP(f)$ and base polyhedron
$\rmB(f)$ of the submodular function $f$ are defined as
\begin{subequations}
	\label{eq:PB}
\begin{align}
	\rmP(f)&:=\Set {x_U\in `R^U\mid x(B)\leq f(B),\forall B\subseteq U} \label{eq:rmP}\\
	\rmB(f)&:=\Set {x_U\in \rmP(f)\mid x(U)=f(U)}, \label{eq:rmB}
\end{align}
\end{subequations}
where $x_U:=(x_i\mid i\in U)$ and $x(B):=\sum_{i\in B} x_i$. (N.b., $\rmP(f)$ and
$\rmB(f)$ are non-empty since $f(`0)\geq 0$.) With $\norm {x_U}$ denoting the Euclidean norm of the
vector $x_U$, the following holds.

\begin{Proposition}[\mbox{\cite{fujishige80,fujishige11}}]
	\label{prop:MNB}
	For any normalized submodular function $f$,
	the minimization
	\begin{align}
		\label{eq:MNB}
		\min \Set{\norm{x_U}\mid x_U\in \rmB(f)}
		%\min_{x_{U}\in \mathbb{R}^{U} } \Set{\norm{x_U} \mid x_U \in \mathbb{R}^{U},  x(B)\leq f(B) \ \forall B\subseteq U, x(U) = f(U)}
	\end{align}
	has a unique solution $x_U^*$, called the minimum (Euclidean) norm base, which satisfies
	\begin{subequations}
	\begin{alignat}{2}
		\kern-1em x^*_i&=\min\Set{`l\in`R\mid i\in  S_{`l}(f)}&\kern1em& \forall i\in U, \text {or equiv.,}\label{eq:MNB:x*}\kern-.5em\\
		\kern-1em S_{`l}(f)&=\Set{i\in U\mid x^*_i\leq `l},&\kern1em & \forall `l\in `R,\label{eq:MNB:S_`l}
	\end{alignat}
	\end{subequations}
	where the equivalence follows directly from Proposition~\ref{prop:ps}.
\end{Proposition}

The minimum norm base, and so the PS by \eqref{eq:MNB:S_`l}, may be computed using Wolfe's minimum norm point algorithm as
in~\cite{fujishige11}. Conversely, by \eqref{eq:MNB:x*},
the minimum norm base can also be computed by any SFM algorithm that solves $S_{`l}$ for any $`l$.
However, the minimum norm point algorithm was shown~\cite{fujishige11} empirically to outperform other SFM algorithms. 

